\chapter{La libreria}

%que
\section{Inicio}
La idea de diseñar y desarrollar un framework en que funcione en el hambiente
de un navegador web, como es Firefox, deja entrever muchos aspectos que no
resultan para nada triviales al momento de codificar.
\begin{itemize}
 \item Se requieren varias lineas de codigo para implementar un framework.
 \item Como llega el codigo al navegador y se inicia su ejecucion.
 \item La cara visible o vista debe ser fasilmente manipulable por la aplicacion
de usuario.
 \item Como los datos generados en el cliente son informados al servidor.
 \item El framework debe brindar soporte a la aplicacion de usuario de una forma
natural y transparente.
 \item Se debe promover al reuso y la extension de funcionalidad del framework.
 \item Como se ponen en marcha los mecanimos o acciones que la aplicacion de
usuario define.
 \item ...
\end{itemize}
En este capitulo se introducen las ideas principales que motivaron la
creacion de una libreria en JavaScript, que brinde el soporte necesario al
framework y a buena parte de los items expuestos.

Si bien el desarrollo de la libreria se mantuvo en paralelo a la del
framework, existen aspectos basicos a los que esta brinda soporte y permiten
presentarla en un apartado separado como una ''Libreria JavaScript'', esta
constituye la base para posteriores construcciones y auna herramientas que
simplifican el desarrollo client-side.

\textit{proto}type + \textit{py}thon = \textit{protopy}

``La creación nace del caos'', la libreria ``Protopy'' no escapa a esta
afirmacion e inicialmente nace de la integracion de Prototype con
las primeras funciones para lograr la modularizacion; con el correr de las
lineas de codigo\footnote{Forma en que los informaticos miden el paso del
tiempo} el desarrollo del framework torna el enfoque inicial poco sustentable,
requiriendo este de funciones mas Python-compatibles se desecha la libreria
base y se continua con un enfoque más ``pythonico'', persiguiendo de esta forma
acercar la semántica de JavaScript 1.7 a la del lenguaje de programacion Python.

No es arbitrario que el navegador sobre el cual corre Protopy sea Firefox y
mas particularmente sobre la version 1.7 de JavaScript. El proyecto mozilla
esta hacercando, con cada nueva versiones del lenguaje, la semantica de JavaScript a
la de Python, incluyendo en esta version generadores e iteradores los cuales
son muy bien explotados por Protopy y el framework.

\subsection*{Protopy}
\textit{Protopy es una libreria JavaScript para el desarrollo de aplicaciones
web dinamicas. Aporta un enfoque modular para la inclusión de código,
orientación a objetos, manejo de AJAX, DOM y eventos.}

Para una referencia completa de la API de Protopy remitase al apandice
\fullref{ch:apendiceProtopy}.

%como
\section{Organizando el codigo}
Undo de los principales inconvenientes a los que Protopy da solucion es a la
inclucion dinamica de funcionalidad bajo demanda, esto se logra con los
``modulos''.
%Forma tradicional vs enfoque de protopy
Tradicionalmente la forma de incluir codigo JavaScript es mediante la
incorporation de una etiqueta ``script'' con un atributo de referencia
al archivo que contiene la funcionalidad. Mas tarde cuando el navegador descarga
el documento y comienza su lectura al encontrar esta etiqueta solicita
al servidor el archivo referenciado y lo interpreta, para continuar luego con la
lectura del resto de las etiquetas. Este enfoque resulta sustentable en el
desarrollo tradicional, en donde el lenguaje brinda mayormente soporte a la
interaccion con el usuario y los fragmentos de codigo que se trasladan al
cliente son bien conocidos por el desarrollador; para un proyecto que implica
gran cantidad de codigo, como en este caso, el enfoque resulta complejo de
sostener.
Buscando una mejor forma de organizar y obtener el codigo es que surge el
concepto de ``modulo'', similar a los modulos en python, cada unidad basica de
de codigo define un modulo, este puede ser importado por otro fragemento de
codigo y cada uno representa su propio ambito de nombres. Existen en Protopy
dos tipos de modulos, los modulos integrados y los modulos organizados en
archivos JavaScript.
%poner mas sobre los archivos js que representan modulos
%Esquema de nombrado
Para acceder a los modulos es necesario establecer un esquema de nombrado \ldots
%Completar sobre esquema de nombres.
Con los modulos y una esquema de nombres para accesarlos, la responsabilidad de
la carga del codigo se deja en manos del cliente y del propio codigo que
requiera determinada funcionalidad provista por un modulo.

Las funciones principales para trabajar con los modulos son ``require'' para
cargar un modulo en el hambito de nombres local y ``publish'' para que los
modulos publiquen o expongan la funcionalidad.

\section{Creando tipos de objeto}
%Semántica de objetos, dentro de la cual se hace una adaptación de
En la programación basada en prototipos las ``clases'' no están presentes, y la
re-utilización de procesos se obtiene a través de la clonación de objetos ya
existentes. 
Protopy agrega el concepto de clases al desarrollo, mediante un contructor de
``tipos de objeto''. De esta forma los objetos pueden ser de dos tipos, las
clases y las instancias. Las clases definen la disposición y la funcionalidad
básicas de los objetos, y las instancias son objetos ``utilizables'' basados en
los patrones de una clase particular.
...

Sets
Diccionarios

%   
%   \begin{itemize}
%   	\item Iteradores
%   	\item Generadores  
%   \end{itemize}
%   \end{item}
%   
%   
%   \item Iteradores
%   \item Generadores

\section{Extendiendo el DOM}
Si bien el \DOM\ ofrece ya una \API\ muy completa para acceder, añadir y cambiar
dinámicamente el contenido estructurado en el documento HTML.

\section{Interactuando con el servidor}

\section{Manejando los eventos}

\section{Envolviendo a gears}

\section{Auditando el codigo}

\section{Ejecutando codigo remoto}

\section{Soporte para json}