El desarrollo de un framework en javascript que funcione del lado del cliente presupone gran cantidad de codigo
viajando de un lado al otro de la conexion. Previendo basicamente este postulado desarrollamos una libreria
que brinde el soporte a los requerimentos mas basicos.
Esta librería contituye la base para posteriores contrucciones mas complejas en el ciente y auna herramientas
que simplifican el desarrollo client-side.

\subsection*{Tipos o clases}
Como ya se menciono anteriormente javascript es un lenguaje orientado a prototipos, para acercarnos un poco a la programacion
de objetos, utilizamos una funcion constructora de tipos o clases a la que denominamos “type”.

var NuevoTipo = type('nombre', bases, [ {} ], {});

\subsection*{Modulos}
Undo de los principales inconvenientes a los que protopy da solucion es a la inclucion dinamica de funcionalidad bajo demanda,
esto es logrado mediante los modulos.
Basicamente un modulo en un archivo con codigo javascript que recide en el servidor y es obtenido y ejecutado en el cliente.

\subsubsection*{Estructura basica de un modulo}
	//modulo.js
	funciones
	objetos
	etc...
	publish(...)

Modulos:
	sys
	exception
	event
	ajax
	builtin
	timer