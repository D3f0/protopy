
\chapter{Protopy}

\section{que}
El desarrollo de un framework en JavaScript funcionando dentro del navegador
Firefox, deja entrever una gran cantidad de detalles que no resultan triviales
al momento de desarrollar.
\begin{itemize}
 \item Se requieren varias lineas de codigo para implementar un framework.
 \item Como llega el codigo al navegador y se inicia su ejecucion.
 \item La cara visible o vista debe ser fasilmente manipulable por la aplicacion
de usuario.
 \item Como los datos generados en el cliente son informados al servidor.
 \item El framework debe brindar soporte a la aplicacion de usuario de una forma
natural y transparente.
 \item Se debe promover al reuso y la extension de funcionalidad del framework.
 \item Como se ponen en marcha los mecanimos o acciones que la aplicacion de
usuario define.
 \item ...
\end{itemize} 
Si bien el desarrollo de Protopy se mantuvo en paralelo a la del
framework, existen aspectos basicos a los que esta brinda soporte y permiten
presentarla en un apartado separado como una ''Libreria JavaScript''. Esta
libreria constituye la base para posteriores construcciones y auna
herramientas que simplifican el desarrollo client-side.

\textit{proto}type + \textit{py}thon = \textit{protopy}

``La creación nace del caos'', Protopy no escapa a esta afirmacion y nace de la
integracion de la libreria Prototype con las primeras funciones para lograr
la modularizacion; con el correr de las lineas de codigo\footnote{Forma en que
los informaticos miden el paso del tiempo} el desarrollo del framework torna
el enfoque inicial poco sustentable, requiriendo este de funciones mas
Python-compatibles se desecha buena parte de la libreria base y se continua con
un enfoque mas ``pythonico''.

En este capitulo se explicara como Protopy da solucion a los items expuestos con
anterioridad. Para una explicacion completa de la API remitace al apandice
\fullref{ch:apendiceProtopy}.

%El desarrollo de un framework en javascript que funcione del lado del cliente
%presupone gran cantidad de codigo
%viajando de un lado al otro de la conexion. Previendo basicamente este
%postulado
%desarrollamos una libreria
%que brinde el soporte a los requerimentos mas basicos.
\subsubsection*{Protopy}
\textit{Protopy es una libreria JavaScript para el desarrollo de aplicaciones
web dinamicas. Aporta un enfoque modular para la inclusión de código,
orientación a objetos, manejo de AJAX, DOM y eventos.}

\section{como}

\section{Organizando el codigo}
Undo de los principales inconvenientes a los que Protopy da solucion es a la
inclucion dinamica de funcionalidad bajo demanda, esto se logra con los ``modulos''.
Tradicionalmente la forma de incluir codigo javascript es mediante la
incorporation de una etiqueta ``script'' con un atributo de referencia
al archivo que contiene la funcionalidad. Mas tarde cuando el navegador descarga
el documento y comienza su lectura al encontrar esta etiqueta solicita
al servidor el archivo referenciado y lo interpreta, para continuar luego con la
lectura del resto de las etiquetas.
Este enfoque es sustentable en el desarrollo tradicional, en donde el lenguaje
brinda mayormente soporte a la interaccion con el usuario y los fragmentos de
codigo a incluir son bien conocidos por el desarrollador.

Basicamente un modulo es un archivo con codigo javascript que recide en el
servidor y es obtenido y ejecutado en el cliente.
...

\section{Creando tipos de objeto}
En la programación basada en prototipos las ``clases'' no están presentes, y la re-utilización de procesos se obtiene a través de la clonación de objetos ya existentes. 
Protopy agrega el concepto de clases al desarrollo, mediante un contructor de ``tipos de objeto''. De esta forma los objetos pueden ser de dos tipos, las clases y las instancias. Las clases definen la disposición y la funcionalidad básicas de los objetos, y las instancias son objetos ``utilizables'' basados en los patrones de una clase particular.
...

Sets
Diccionarios

\section{Manipulando el DOM}

\section{Interactuando con el servidor}

\section{Manejando los eventos}

\section{Emvolviendo a gears}

\section{Auditando el codigo}

\section{Ejecutando codigo remoto}

\section{Soporte para json}