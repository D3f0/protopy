\chapter{Propuesta de framework MTV en el cliente}

\section{que}
El desarrollo de un framework en JavaScript funcionando dentro del navegador
Firefox, deja entrever una gran cantidad de detalles que no resultan triviales
al momento de desarrollar.
\begin{itemize}
 \item Se requieren varias lineas de codigo para implementar un framework.
 \item Como llega el codigo al navegador y se inicia su ejecucion.
 \item La cara visible o vista debe ser fasilmente manipulable por la aplicacion
de usuario.
 \item Como los datos generados en el cliente son informados al servidor.
 \item El framework debe brindar soporte a la aplicacion de usuario de una forma
natural y transparente.
 \item Se debe promover al reuso y la extension de funcionalidad del framework.
 \item Como se ponen en marcha los mecanimos o acciones que la aplicacion de
usuario define.
 \item ...
\end{itemize} 
Si bien el desarrollo de Protopy se mantuvo en paralelo a la del
framework, existen aspectos basicos a los que esta brinda soporte y permiten
presentarla en un apartado separado como una ''Libreria JavaScript''.

\textit{proto}type + \textit{py}thon = \textit{protopy}

``La creación nace del caos'', Protopy no escapa a esta afirmacion y nace de la
integracion de la libreria Prototype con las primeras funciones para lograr
la modularizacion; con el correr de las lineas de codigo\footnote{Forma en que
los informaticos miden el paso del tiempo} el desarrollo del framework convierte
el enfoque inicial en poco sustentable, requiriendo este de funciones mas
Python-compatibles se desecha buena parte de la libreria base y se continua con
un enfoque mas ``pythonico''.

En este capitulo se explicara como Protopy da solucion a los items expuestos y
terminando una definicion.

\textit{Protopy es una libreria JavaScript para el desarrollo de aplicaciones
web dinamicas. Aporta un enfoque modular para la inclusión de código,
orientación a objetos, manejo de AJAX, DOM y eventos.}

% llamaron para papa, llamar a este numero 0800-666-1153 rio 
\section{como}

El desarrollo de un framework en javascript que funcione del lado del cliente
presupone gran cantidad de codigo
viajando de un lado al otro de la conexion. Previendo basicamente este postulado
desarrollamos una libreria
que brinde el soporte a los requerimentos mas basicos.
Esta librería contituye la base para posteriores contrucciones mas complejas en
el ciente y auna herramientas
que simplifican el desarrollo client-side.


\section{Modulos}
Undo de los principales inconvenientes a los que protopy da solucion es a la
inclucion dinamica de funcionalidad bajo demanda,
esto es logrado mediante los modulos.
Basicamente un modulo en un archivo con codigo javascript que recide en el
servidor y es obtenido y ejecutado en el cliente.