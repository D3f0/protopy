%%
%% This is file `sampleNtn2.tex',
%% generated with the docstrip utility.
%%
%% The original source files were:
%%
%% glossary.dtx  (with options: `sampleNtn2.tex,package')
%% Copyright (C) 2006 Nicola Talbot, all rights reserved.
%% If you modify this file, you must change its name first.
%% You are NOT ALLOWED to distribute this file alone. You are NOT
%% ALLOWED to take money for the distribution or use of either this
%% file or a changed version, except for a nominal charge for copying
%% etc.
%% \CharacterTable
%%  {Upper-case    \A\B\C\D\E\F\G\H\I\J\K\L\M\N\O\P\Q\R\S\T\U\V\W\X\Y\Z
%%   Lower-case    \a\b\c\d\e\f\g\h\i\j\k\l\m\n\o\p\q\r\s\t\u\v\w\x\y\z
%%   Digits        \0\1\2\3\4\5\6\7\8\9
%%   Exclamation   \!     Double quote  \"     Hash (number) \#
%%   Dollar        \$     Percent       \%     Ampersand     \&
%%   Acute accent  \'     Left paren    \(     Right paren   \)
%%   Asterisk      \*     Plus          \+     Comma         \,
%%   Minus         \-     Point         \.     Solidus       \/
%%   Colon         \:     Semicolon     \;     Less than     \<
%%   Equals        \=     Greater than  \>     Question mark \?
%%   Commercial at \@     Left bracket  \[     Backslash     \\
%%   Right bracket \]     Circumflex    \^     Underscore    \_
%%   Grave accent  \`     Left brace    \{     Vertical bar  \|
%%   Right brace   \}     Tilde         \~}
\documentclass{report}

\usepackage[plainpages=false,colorlinks]{hyperref}
\usepackage{html}
\usepackage[toc]{glossary}[2006/07/20]

\makeglossary

\newglossarytype[nlg]{notation}{not}{ntn}[number=none]

\newcommand{\notationname}{Notation}

\makenotation

\begin{document}

\storeglosentry[notation]{not:set}{name={$\mathcal{S}$},
description=A set,sort=S}

\storeglosentry[notation]{not:U}{name=$\mathcal{U}$,
description={The universal set},sort=U}

\storeglosentry[notation]{not:card}{name=$\mid\mathcal{S}\mid$,
description=cardinality of $\mathcal{S}$,
sort=cardinality}

\storeglosentry{glos:set}{name=set,description=A collection of
distinct objects}

\storeglosentry{glos:card}{name=cardinality,description=The number of
elements in the specified set}

\title{Sample Document using glossary Package}
\author{Nicola Talbot}
\pagenumbering{alph}
\maketitle

\begin{abstract}
\pagenumbering{Alph}
This is a sample document illustrating the use of the
\textsf{glossary} package.  In this example, a new glossary type
called \texttt{notation} is defined, so that the document can have a
separate glossary of terms and index of notation. The index of notation
doesn't have associated numbers.
\end{abstract}

\pagenumbering{roman}
\tableofcontents

\printglossary


\printnotation

\chapter{Introduction}
\pagenumbering{arabic}

Sets\useglosentry{glos:set}
are denoted by a caligraphic font
e.g.\ $\mathcal{S}$\useglosentry{not:set}.

Let \gls{not:U} denote the universal set.

The \gls{glos:card} of a set $\mathcal{S}$ is denoted
\gls{not:card}.

\chapter{Another Chapter}

Another mention of the universal set \gls{not:U}.

\end{document}
\endinput
%%
%% End of file `sampleNtn2.tex'.
