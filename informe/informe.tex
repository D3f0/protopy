\documentclass[a4paper]{report}

\usepackage[utf8]{inputenc}
\usepackage[spanish]{babel}


\begin{document}

\title{Sistemas Web Desconectados}
\author{Defossé Nahuel, van Haaster Diego Marcos}
\date

\maketitle

\pagenumbering{roman}
\tableofcontents
\listoffigures
\listoftables

\chapter*{Agradecimientos}

\begin{abstract}

\end{abstract}

\pagenumbering{arabic}

\chapter{Introducción}
\label{ch:intro}

\chapter{Motivación}
\label{ch:motiv}
Hoy en día Internet supone un excelente medio para obtener información sobre diversos temas. 
Para que esta información sea realmente útil es imprescindible que el acceso a ella sea simple
e intuitivo, de forma que cualquier persona pueda encontrar y utilizar lo que desea 
con tan sólo unos conocimientos básicos.
\section{Un poco de historia sobre la WWW}
Esto es posible gracias a la world wide web (Web), la cual surge alrededor  del año 1990 
en el CERN\footnote{Organización Europea para la Investigación Nuclear},
ante la necesidad de distribuir e intercambiar información acerca de investigaciones 
de una manera más efectiva. "Tim" John Berners-Lee fue quien definió los componentes 
básicos que constituyen la Web y dan soporte al almacenamiento, transmisión e 
interpretación de los documentos de hipertexto.

El término hipertexto no resultaba nuevo en aquel momento: fue acuñado por Ted Nelson 
en 1965, para designar a los documentos que pueden contener enlaces
(referencias) a otras partes del documento o a otros documentos. De esta forma, 
el documento no necesita ser leído secuencialmente. El hipertexto dio un gran 
salto con el desarrollo de Internet, posibilitando que un documento este
físicamente distribuido en distintas máquinas conectadas entre sí.

Berners-Lee desarrolló lo que por sus siglas en inglés se denominan: el
lenguaje HTML\footnote{HyperText Markup Language} o lenguaje de etiquetas de
hipertexto, el protocolo HTTP\footnote{HyperText Transfer Protocol} y el sistema
de localización de objetos en la Web URL\footnote{Universal Resource
Locator \url{http://es.wikipeida.com/url}{URL en la Wikipeida}}, en conjunto
con las herramientas necesarias para que la Web funcionase: el primer navegador y
editor Web, el primer servidor Web y las primeras páginas Web, 
que al mismo tiempo describían el proyecto.

La Web funciona siguiendo el denominado modelo cliente-servidor, 
habitual en las aplicaciones que funcionan en una red: existe un servidor, 
que es quien presta el servicio, y un cliente, que es quien lo recibe.
El cliente web o navegador es un programa con el que el usuario interacciona 
para solicitar a un servidor web el envío de documentos codificados en
lenguaje HTML. Estos documentos se transfieren mediante el protocolo HTTP. El 
navegador debe interpretar el HTML para mostrar la información al usuario en el formato adecuado.
El servidor web, o simplemente servidor, es un programa que está permanentemente
escuchando las peticiones de conexión de los clientes mediante el protocolo
HTTP. Cuando llega una petición, el servidor busca el documento que ha sido
solicitado en su sistema de archivos y, si lo encuentra, lo envía al cliente;
en caso contrario, devuelve un error.

Con el paso de los años se fueron incorporando innovaciones en el servidor. El 
siguiente paso lógico lo constituyeron los documentos o páginas dinámicas. 
Éstas se generan al ser solicitadas con información específica del momento o 
del usuario. Las aplicaciones CGI (Common Gateway Interface) surgieron como una
de las primeras maneras prácticas de crear el contenido de una forma dinámica. 
En una aplicación CGI, el servidor pasa las solicitudes del cliente a un programa 
externo. Este programa puede estar escrito en cualquier lenguaje que el servidor 
soporte e interactuar con bases de datos u otros recursos que el servidor posea. 
Su salida se envía al cliente, en lugar del tradicional archivo estático.

El uso extendido de CGI evolucionó paulatinamente hacia el diseño e 
implementación de frameworks. Un framework representa una arquitectura de software que
modela las relaciones generales de las entidades del dominio. Provee una estructura
y una metodología de trabajo. Típicamente, un framework puede incluir soporte de 
base de datos, programas, librerías, entre otros, para ayudar a desarrollar y 
vincular los diferentes componentes de un proyecto.

Paralelamente a las innovaciones que surgían en el servidor, se comenzaron a 
agregar en los navegadores nuevas funcionalidades, entre las que se destacan los 
intérpretes para lenguajes de scripting. JavaScript es uno de estos lenguajes; 
paulatinamente tomó relevancia y se convirtió en un estándar. Se ejecuta en 
el navegador al mismo tiempo que se descarga junto con el código HTML y permite 
una modificación o interacción con el código de la página a través del manejo del 
DOM (Document Object Model). Netscape incorporo por primera vez a DOM, con el 
fin de acceder, añadir y cambiar dinámicamente contenido estructurado en una página; 
esto se denomina HTML dinámico (DHTML).


Los avances, tanto en el servidor como en el navegador, convergieron en el nacimiento
de un nuevo concepto: las "aplicaciones web". En una aplicación web las páginas son 
generadas dinámicamente en un formato estándar soportado por los navegadores. 
Agregando lenguajes interpretados en el lado del cliente, se añaden elementos 
dinámicos a la interfaz de usuario. Generalmente, cada página web en particular 
se envía al cliente como un documento estático, pero la secuencia de páginas 
ofrece al usuario una experiencia interactiva. Durante la sesión, el navegador 
interpreta y muestra en pantalla las páginas, actuando como cliente para cualquier aplicación web.

Las aplicaciones web son populares debido a lo práctico que resulta el navegador 
web como cliente de acceso a las mismas. También resulta fácil actualizar y mantener 
aplicaciones web sin distribuir e instalar software a miles de usuarios potenciales. 
En la actualidad, existe una gran oferta de frameworks web para facilitar el 
desarrollo de aplicaciones web.
   
Una ventaja significativa de las aplicaciones web es que funcionan 
independientemente de la versión del sistema operativo instalado en el cliente. 
En vez de crear clientes para los múltiples sistemas operativos, la aplicación web 
se escribe una vez y se ejecuta igual en todas partes.
    
Las aplicaciones web tienen ciertas limitaciones en las funcionalidades que 
ofrecen al usuario. Hay funcionalidades comunes en las aplicaciones de escritorio, 
como dibujar en la pantalla o arrastrar y soltar, que no están soportadas por 
las tecnologías web estándar. Los desarrolladores web, generalmente, utilizan 
lenguajes interpretados o script en el lado del cliente para añadir más 
funcionalidades, especialmente para ofrecer una experiencia interactiva que 
no requiera recargar la página cada vez. Recientemente se han desarrollado 
tecnologías para coordinar estos lenguajes con tecnologías en el lado del servidor.
El alcance universal de la Web la ha hecho un terreno muy atractivo para la 
implementación de sistemas de información. Los sistemas operativos actuales de 
propósito general cuentan con un navegador web, con posibilidades de acceso a 
bases de datos y almacenamiento de código y recursos.
    
    
La web, en el ámbito del software, es un medio singular por su ubicuidad y 
sus estándares abiertos. El conjunto de normas que rigen la forma en 
que se generan y transmiten los documentos a través de la web son 
regulados por la W3C (Consorcio World Wide Web). La mayor parte de la web está 
soportada sobre sistemas operativos y software de servidor que se rigen bajo 
licencias OpenSource1 (Apache, BIND, Linux, OpenBSD, FreeBSD). Los lenguajes 
con los que son desarrolladas las aplicaciones web son generalmente OpenSource, 
como e PHP, Python, Ruby, Perl y Java. Los frameworks web escritos sobre estos 
lenguajes utilizan alguna licencia OpenSource para su distribución; incluso 
frameworks basados en lenguajes propietarios son liberados bajo licencias OpenSource.


\chapter{Propuesta}
\label{ch:propuesta}
Las principales empresas de software se están abocando a desarrollar aplicaciones 
que funcionan tanto conectadas a la Web como desconectadas de ella, brindando 
soluciones que reúnan lo mejor de las dos opciones, como lectores de noticias, 
sistemas de gestión de proyectos y hasta paquetes de oficina (procesadores de 
texto, hojas de cálculo, etc.), por mencionar algunas.

A mediados del 2007 Google libera un complemento para los navegadores web 
llamado \emph{Google Gears}, que provee un servidor web de contenido estático, 
una base de datos y un mecanismo para ejecutar tareas en segundo plano, 
llamado Worker Pool. Mediante este complemento se permite almacenar localmente
los documentos y otros elementos, como las imágenes y el código JavaScript, 
presentando como novedad la capacidad de almacenar datos de la aplicación web 
en el cliente, en una base de datos de similares características a las que se
encuentran normalmente en los servidores web. Varias aplicaciones de Google 
comienzan a hacer uso de este complemento, siendo notable su utilización en 
la aplicación de oficina basada en web, Google Docs (http://docs.google.com), 
la cual brinda un procesador de texto y una planilla de cálculo que pueden ser 
usadas incluso cuando el cliente está desconectado de la web, con la condición 
previa de que haya entrado una vez al sitio y habilitado el modo desconectado.

En la actualidad existen alrededor de 150 frameworks de aplicaciones web 
(desarrollados en los lenguajes PHP, Python, Ruby, Perl, Lua, ASP, Java, 
ColdFusion, Groovy y Common LISP), que siguen en mayor o menor grado el
patrón Modelo-Vista-Controlador (MVC), este es un patrón de arquitectura de
software que separa los datos de una aplicación, la interfaz de usuario, 
y la lógica de control en tres componentes distintos. 


\chapter{Resultados}
\label{ch:results}
WIP\footnote{Work in progress :)}


\chapter{Plan de trabajo}
\label{ch:milestones}


El framework sobre el cual se instrumentará esta funcionalidad, será elegido tras 
un análisis de su modelo MVC y la facilidad de desarrollo.

En cuanto a la transferencia de lógica y de datos, se intentará realizar un
transferencia de la lógica de negocio de la aplicación web al cliente, 
de manera de evitar redundancia. Es decir, una vez definido el modelo de datos, 
éste se adaptará para que pueda ser manejado por el cliente de manera automática.

Esta transferencia del modelo estará sujeto a criterios de seguridad, privacidad
y consistencia de de datos que serán añadidos como atributos adicionales al 
modelo existente en la aplicación web.

También será importante brindar mecanismos para la sincronización de la
aplicación desconectada con la aplicación web.

Se espera poder brindar un sistema de migraciones, de manera que la aplicación 
desconectada pueda adaptarse a los cambios en el modelo de datos en el servidor.


\section{Primera etapa}
\subsection{Indagación bibliográfica}
\subsection{Selección de framework web}
\begin{itemize}
  \item Definición de las aplicaciones web, su estructura y su mecánica de
  trabajo, identificando las partes que generan transformación 
  de la información en contraste con las aplicaciones tradicionales.
  \item Identificación de condiciones y requisitos necesarios para que una
  aplicación web se ejecute en forma desconectada a la red que la provee.
  \item Establecimiento de las capacidades y las opciones que existen
  actualmente para dar soporte a una aplicación desconectada.
  \item Identificación de los componentes de un framework web.
  \item Identificación de los frameworks MVC. Generalidades sobre las opciones
  más populares.
  \item Generalidades sobre el modelo de desarrollo de los framework web.
  \item Evaluación de un conjuto de frameworks MVC.
\end{itemize}

\subsection{Diseño e implementación}
	
\subsubsection{Generación de contenido a partir de un framework}
	
\begin{itemize}
	\item Generación de contenido "estático" utilizando los componentes de un framework MVC.
	\item Generación de contenido dinámico de ejecución retardada en el cliente.
	\item Almacenamiento de datos en el cliente y diferentes mecanismos de
	transporte.
	\item Organización de datos y meta información sobre el modelo de dominio.
	\item Delegación de responsabilidad al cliente. Ejecución concurrente.
	\item Captura de eventos de la interfase en modo desconectado.
	\item Consideraciones sobre degradación de servicio.
\end{itemize}            
            
\subsubsection{Desconexión}
\begin{itemize}
  \item Sincronización basada en marcas de tiempo, privilegios y prioridades.
  \item Delegación del control de requisiciones web a rutina totalmente
  manejada por el cliente.
  \item Rutina de puente de contenido dinámico/estático.
  \item Rutinas de intercepción de flujo de control o controlador del lado del
  cliente.
\end{itemize}

\subsubsection{Interrelación con la aplicación}

\begin{itemize}
  \item Esquema de autorización y privilegios sobre los datos que son enviados
  al cliente.
  \item Generación de esquemas de reflexión de las entidades del modelo de
  negocios de la aplicación online y sus contrapartes offline.
  \item Agregación de meta información para la generación de una vista de datos
  para lograr Seguridad.
  \item Provisión de garantías de integridad de datos.
  \item Consideraciones sobre seguridad.
  \item Consideraciones sobre políticas de sincronización.
  \item Consideraciones sobre la evolución del modelo de datos: migraciones.
\end{itemize}

\subsection{Ejemplo de aplicación}
\subsection{Conclusiones y posibles extensiones}

\chapter{Conclusiones}
\label{ch:conc}
\begin{equation}
\int_{a=0}^{b=3}{\{
 \sum_{ahora}^{9/3/2009}{ trabajo\ de\ diego }\ +\ \sum_{9/3/2009}^
 {fines\ de\ abril}{aportes\ de\ nahuel}\}}
\end{equation}

%\bibliographystyle{plain}
%\bibliography{thesis}

\begin{thebibliography}{99} % No se por que hay que poner el número de libros
  % eje,plo 
  %\bibitem[label1]{cite_key1} bibliographic information
  %\bibitem[label2]{cite_key2} bibliographic information
  \bibitem{W3C}{Tratado de la W3C sobre las aplicaciones web desconectadas.
  http://www.w3.org/TR/offline-webapps/}

  \bibitem{offline_web_aps}{Llevando las aplicaciones basadas en web al ámbito
  offline. http://www.linux.com/feature/60827/} 

  \bibitem{kevin_lynch_interv}{Entrevista con Kevin Lynch sobre el futuro de las
  aplicaciones web desconectadas.
  http://www.technologyreview.com/read_article.aspx\?ch=specialsections\&sc\=emerging08\&id=20245}

  \bibitem{Robert_O_Callahan}{Well, I'm Back Robert O'Callahan. 
	Aplicaciones web desconectadas.
	http://weblogs.mozillazine.org/roc/archives/2007/02/offline_web_app.html} 

  \bibitem{google_gears_docs}{Documentación para desarroladores de Google Gears. 
  http://code.google.com/apis/gears/}

  \bibitem {tim_barners_lee} {Biografía de Tim Barners-Lee. 
  Wikipedia http://es.wikipedia.org/wiki/Tim_Berners-Lee}

  \bibitem{wiki_web}{Definición de Web, Wikipedia.
  http://es.wikipedia.org/wiki/World_Wide_Web}

  \bibitem{wiki_http}{}Definición de Protocolo HTTP, Wikipedia.
  http://es.wikipedia.org/wiki/HTTP}

  \bibitem{wiki_js}{Definición de JavaScript, Wikipedia.
  http://es.wikipedia.org/wiki/JavaScript}

  \bibitem{wiki_cgi}{Definición de CGI, Wikipedia.
  http://es.wikipedia.org/wiki/Cgi}

  \bibitem{wiki_dhtml}{Definición de DHTML, Wikipedia.
  http://es.wikipedia.org/wiki/DHTML}

  \bibitem{wiki_web_framework}{Definición de Framework Web, Wikipedia.
  http://en.wikipedia.org/wiki/Web_application_framework}

  \bibitem{wiki_web_frameworks_list}{Listado de frameworks web, Wikipedia. 
  http://en.wikipedia.org/wiki/List_of_web_application_frameworks}

\end{thebibliography}


\end{document}
