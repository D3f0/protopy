\documentclass[a4paper]{report}

\usepackage[utf8]{inputenc}
\usepackage[spanish]{babel}
\usepackage[bookmarks=false,colorlinks=true,linkcolor={blue},pdfstartview={XYZ null null 1.22}]{hyperref}

\begin{document}

\title{Sistemas Web Desconectados}
\author{Defossé Nahuel, van Haaster Diego Marcos}
\date{\today}

\maketitle

\pagenumbering{roman}
\tableofcontents
\listoffigures
\listoftables

\chapter*{Agradecimientos}

\begin{abstract}

\end{abstract}

\pagenumbering{arabic}

\chapter{Introducción}
\label{ch:intro}

\chapter{Capítulo teórico}
\section{Django}
\href{http://www.djangoproject.com}{Django} es un framework web escrito en
Python\footnote{\href{http://www.python.org/}{Lenguaje de programación
orientado a objetos multiparadigma}} el cual sigue vagamente el concepto de Modelo Vista
Controlador. Ideado inicialmente como un adminsitrador de contenido para varios sitios de noticias, los desarrolladores encontraron que su CMS era lo sufcientemente genérico como
para curbir un ámbito más aplio de aplicaciones. Fue liberado\footnote{En el
ámbito del software libre, la liberación es la fecha en la cual se pone a
disposición de la comunidad del software en cuestión} bajo la licencia BSD en
Julio del 2005 como Django Web Framework en honor a Django Reinhart. En junio del
2008 fue anuncidada la cereación de la Django Software Fundation, la cual se hace
cargo hasta la fecha del desarrollo y mantenimiento.

Django implementa una estructura \emph{modelo}, \emph{vista}, \emph{plantilla}.
O MTV por sus siglas en inglés (model, view, template).
\begin{description}
\item[Modelo] Modelo de datos de la aplicación. Consiste en un mapeador Objeto
Relacional.
\item[Vista] Funciones a ser ejecutadas cuando un cliente accede a la
aplicación.
\item[Template] La generación de salida (típicamente código HTML)
\end{description}




\chapter{Plataforma}
\section{Plataforma Mozilla}
\begin{itemize}
  \item Porque desarrollaron e implementaron Javascript 1.7
  \item Porque javascript 1.7 tomo semántica (y sintaxis???) de Python
  \item Porque es código abierto
  \begin{item}
  	Porque es extensible mediante plugins
  	\begin{itemize}
        \item Tiene firebug
        \item Gears y firebug = muy compardor para el desarrollador.
     \end{itemize}
      
  \end{item}
  \item 
\end{itemize}



\chapter{Referencia Protopy}
\label{ch:protopy}

Protopy persigue acercar la semántica del Javascript 1.7 a la del lenguaje
Python. Las funcionalidades principales son las siguientes:
\begin{itemize}
  \item Ámbito de nombres
  \begin{item}
  Semántica de objetos, dentro de la cual se hace una adaptación de
  \begin{itemize}
  	\item Iteradores
  	\item Generadores  
  \end{itemize}
  \end{item}
  
  
  \item Iteradores
  \item Generadores
  
  \item Tipos básicos de la librería estándar de python, entre los que se
  encuentran: $ bool, $
\end{itemize}


\begin{description}
\item[type]{
Type sirve para defninir clases.
}
\end{description}
\begin{thebibliography}{99}

	\bibitem{javascript_versions}{\href{http://ejohn.org/blog/versions-of-javascript/}
		{Versiones de Javascript, Jhon Reisig}
	}

\end{thebibliography}
\end{document}